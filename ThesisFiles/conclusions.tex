\chapter{总结}

本文总结如下
\begin{itemize}
    \item 对于 DFA 中的陷阱状态,可以使用函数 “DFA::usefulf()” 去除;
    \item 除了算法 DFA::min\_Brzozowski() ,其他算法都不能去除 DFA 中的开始不可达状态;
    \item 由于算法 DFA::min\_Brzozowski() 在某些情况下,会改变最小的 DFA 接受的语言,所以本文认为 FIRE engine 中对 Brzozowski 算法的实现还有不足的地方;
    \item 对函数 DFA::usefulf() 的修改有较高可信度,详见第 \ref{sec:usefulf} 节;
    \item 算法 DFA::min\_Hopcroft() 在多个样例中输出与预期不符,本文认为 FIRE engine 对 Hopcroft 算法的实现仍有不足;
    \item 经过测试,算法 DFA::min\_HopcroftUllman() 、 DFA::min\_dragon() 和 DFA::min\_Watson() 表现符合预期情况,本文认为这三个算法可以用于验证其他用于 DFA 最小化算法的正确性和有效性。
\end{itemize}


% DFA::min\_Brzozowski()   
% DFA::min\_Hopcroft()
% DFA::min\_HopcroftUllman()   
% DFA::min\_dragon()
% DFA::min\_Watson()