\chapter{总结}

本文总结如下
\begin{itemize}
    \item 对于 DFA 中的陷阱状态,可以使用函数 “DFA::usefulf()” 去除;
    \item 除了算法 DFA::min\_Brzozowski() ,其他算法都不能去除 DFA 中的开始不可达状态;
    \item 由于算法 DFA::min\_Brzozowski() 在某些情况下,会改变最小的 DFA 接受的语言,所以本文认为 FIRE engine 中对 Brzozowski 算法的实现还有不足的地方;
    \item 对函数 DFA::usefulf() 的修改有较高可信度,详见第 \ref{sec:usefulf} 节;
    \item 算法 DFA::min\_Hopcroft() 严格来说不是算法的错误,而是数据不符合其要求,Bruce William Watson 在他的论文\cite[第三节]{watson1993taxonomyb}中基于完全 DFA 讨论了状态等价和最小化的相关内容原文为“ In this subsection, we restrict ourselves to considering minimization of $Compelete$ DFA's ”,但是在进行论文\cite[第四节]{watson1993taxonomyb}中描述最小化算法时没有提及限定在完全 DFA 上,再加之除了 Hopcroft 的算法(算法\ref{al:4-8}),其他算法并没有输出错误结果,所以对判断算法的正确性有一定的误导性。对于此算法的运行结果,本文认为第 \ref{subsec:solve-hopcroft} 节中的方案 2 ,也即构造完全自动机之后再进行最小化为比较合适的解决方法;
    \item 经过测试,算法 DFA::min\_HopcroftUllman() 、 DFA::min\_dragon() 和 DFA::min\_Watson() 表现符合预期情况,本文认为这三个算法可以用于验证其他用于 DFA 最小化算法的正确性和有效性。
    \item 本文也 FIRE engine 新增加了函数 “DTransRel::closure()” ,用于实现 $SReachable$。新增了函数 “DFA::usefulfs()”,用于移除 DFA 中的开始不可达状态。新增函数 “DFA::complete()”,用于构造一个完全 DFA。新增函数 “DFA::Complete()” ,用于检查 一个 DFA 是否是完全 DFA。
\end{itemize}


% DFA::min\_Brzozowski()   
% DFA::min\_Hopcroft()
% DFA::min\_HopcroftUllman()   
% DFA::min\_dragon()
% DFA::min\_Watson()