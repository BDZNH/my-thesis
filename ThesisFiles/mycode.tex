%!TEX root = ../Demo.tex
\chapter{代码}

一个实例化 DFA 类的例子如代码 \ref{lst:DFASample} 所示。代码 \ref{lst:DFASample} 中执行了 DFA::usefulf() 函数,去除多余的状态,让输出数据与原始数据相对应。代码 \ref{lst:DFASample} 对应的 DFA 为图 \ref{fig:keepMin-1-nonTheState} 。


\lstset{style=mystyle}
\begin{lstlisting}[language=C++,label={lst:DFASample},caption={实例化DFA示例}]
#include"DFA.h"
#include<iostream>
int main()
{
    DFA_components dfa_com1;

    // StateSet S  开始状态集
    dfa_com1.S.set_domain(10);
    dfa_com1.S.add(0);

    // StateSet F  结束状态集
    dfa_com1.F.set_domain(10);
    dfa_com1.F.add(3);

    // StatePool Q 
    int i = 10;
    while (i--)
    {
        dfa_com1.Q.allocate();
    }

    // DTransRel T transition             
    dfa_com1.T.set_domain(10);
    dfa_com1.T.add_transition(0, '0', 1);
    dfa_com1.T.add_transition(1, '0', 2);
    dfa_com1.T.add_transition(2, '0', 3);
    dfa_com1.T.add_transition(3, '0', 3);
    dfa_com1.T.add_transition(0, '1', 0);
    dfa_com1.T.add_transition(1, '1', 0);
    dfa_com1.T.add_transition(2, '1', 0);
    dfa_com1.T.add_transition(3, '1', 0);

    DFA dfa1(dfa_com1);
    dfa1.usefulf();

    std::cout<<dfa1<<std::endl;
	
    return 0;
}
\end{lstlisting}