%!TEX root = ../Demo.tex
% 中文摘要
\begin{abstract}
  本文对有限自动机 C++ 工具箱中的五个最小化算法进行功能测试。有限自动机 C++ 工具箱对 Hopcroft 算法(由 Hopcroft 提出)的实现需要输入数据为完全的确定性有限自动机。而对于其他三个最小化算法 HopcroftUllman 算法(由 Hopcroft 和 Ullman 提出)、 dragon 算法(由 Aho Sethi 和 Ullman 提出),Waston 算法(由 Bruce William Watson 提出)则没有明确要求数据数据是否是为完全的确定性有限自动机。除了 Brzozowski 的算法外,以上四个最小化算法都依赖于计算状态的等价关系(或可区分的关系),有限自动机 C++ 工具箱对 Brzozowski 算法的实现有一定的缺陷,在某些情况下会输出错误的结果。

  本文还为有限自动机 C++ 工具箱增加了用来计算从开始状态可以到达的状态的集合的算法(SReachable)、移除有限自动机中从开始状态不可到达的状态的算法(usefuls)和构造完全自动机的算法(complete)。

\end{abstract}
\keywords{确定性有限自动机, 最小化, 算法, 状态等价}

% 英文摘要
\begin{enabstract}
  This paper is just a sample example for the users in learning the \XDUthesis. I will try my best to use the commands and environments which are involved by the \XDUthesis. Also, the popular composition skills in figures, tables and equations will be elaborated.
  
  In the part unimportant, I will show something others, such as poems and lyrics.

\end{enabstract}
\enkeywords{XDUthesis, commands, environments, skills}