%!TEX root = ../Demo.tex
\chapter{引言}
自动机理论是许多科学的重要理论基础,从硬件电路的简化,到各式各样的编译器构造,到处都有着自动机理论的应用,从自动机理论的诞生开始,自动机的最小化就一直是一个重要的关注点,因为自动机的状态数约少,意味着它使用的资源也会越少,这一点对资源敏感的自动机应用很重要。已经有很多人为此做了大量的工作。1954 年,Huffman 提出了经典的用于确定性有限自动机的的最小化算法\cite{HUFFMAN1954161},随后更高效的最小化算法由 Hopcroft 和 Moore 提出。确定性有限自动机(Deterministic Finite Automata,DFA)的最小化是自动机理论的一个重要组成部分,研究确定性有限自动机的最小化对自动机理论的健全和发展有重要意义。

\section{FIRE engine}
FIRE engine \cite{watson1994design}是一个使用C++语言实现有限自动机和正则表达式的类库。本文中也把“有限自动机C++工具箱”称作 “FIRE engine” 。有限自动机C++工具箱(FIRE engine)实现了论文 \cite{watson1993taxonomya,watson1993taxonomyb} 中提及的大部分算法,其中也包括了 Hopcroft、Ullman 等人提出的著名的确定性有限自动机最小化算法,验证这些算法的正确性和有效性对研究其他最小化算法很有帮助。

\section{国内外研究现状}

自动机理论的发展已经有很长时间,有限状态自动机、确定性有限状态自动机及相关的理论出现的时间都比较早。国外已经有文章 \cite{watson1993taxonomyb} 对一些著名的确定性有限自动机最小化算法,如 Hopcroft、Brzozowski、Ullman 等提出的最小化算法和这些算法之间的演化关系做出了详尽的阐述。

据称 Hopcroft 的算法有着最优的时间复杂度, 此算法时间复杂度为 $\mathcal{O}$($n$ log $n$)\cite{Hopc71}。严格来说,Hopcroft 的算法的时间复杂度不仅仅与自动机的大小 $n$ \footnote{ $n$ 为自动机的状态数,也称自动机的大小。}有关,还与自动机的输入的字母表的大小 $k$ 有关,其时间复杂度应为$\mathcal{O}$($kn$ log $n$)。Timo Knuutila 指出,在自动机的输入的字母表的大小不固定时,Hopcroft 的原始算法的时间复杂度将不再是$\mathcal{O}$($kn$ log $n$)\cite{KNUUTILA2001333}。

以状态的等价为基础,通过分层逼近、逐点近似等方法来计算等价关系,并在此基础上衍生处其他最小化算法。文中提及的算法中,Brzozowski 的最小化算法不依赖于其他任何最小化算法,也不依赖于确定状态的等价,而其他著名的算法的基础都是状态等价,是在已有的算法上的改进\cite{watson1993taxonomyb}。

国内对最小化算法的研究则针对一些更加详细的分类。比如在胡芙提出的超最小化算法,用来把确信性树自动机转换为确信性有限自动机,然后用等价状态合并和等价类分割,最后得到差异有限的确定性树自动机\cite{hfTreeFA}。还有张锋的基于 Myhill-Nerode 原理推导出来的针对直觉模糊有限自动机和完备格上的直觉模糊有限自动机的最小化算法\cite{zf14Min}。对于自动机技术的实际应用,孙丹丹在海量的 Web 信息中抽取用户需要的信息上应用了模糊树自动机技术,并且提出模糊树自动机的构造算法和最小化算法\cite{sdd14}。 

\section{本文的主要工作}